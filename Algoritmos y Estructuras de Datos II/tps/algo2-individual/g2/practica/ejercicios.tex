\documentclass[10pt, a4paper]{article}
\usepackage{enumerate, listings, enumitem, etoolbox}
\usepackage[spanish]{babel}
%\usepackage{a4wide} % márgenes un poco más anchos que lo usual
\usepackage{fullpage}
\usepackage[utf8]{inputenc}
\usepackage{aed2-symb,aed2-itef,aed2-tad,aed2-diseno}
\usepackage{caratula}

\newcommand\impluego{\opsub{\Rightarrow}{\scriptscriptstyle{L}}}
\usepackage{color} % para snipets de codigo coloreados
\usepackage{fancybox}  % para el sbox de los snipets de codigo

\definecolor{litegrey}{gray}{0.94}
\newenvironment{codesnippet}{%
	\begin{Sbox}\begin{minipage}{\textwidth}\sffamily\small}%
	{\end{minipage}\end{Sbox}%
		\begin{center}%
		\vspace{-0.4cm}\colorbox{litegrey}{\TheSbox}\end{center}\vspace{0.3cm}}

% \AtBeginEnvironment{align}{\setcounter{equation}{0}}
\parskip=5pt % 10pt es el tamaño de fuente
\parindent=0pt

\begin{document}

\titulo{Guia 2}
\subtitulo{Ejercicios obligatorios de la pr\'actica}

\fecha{\today}

\materia{Algoritmos y Estructuras de Datos II}

\integrante{Rodriguez, Miguel}{57/19}{mmiguerodriguez@gmail.com}

\maketitle

\textbf{Ejercicio 1: Funciones polinomiales vs exponenciales}
\begin{enumerate}[label=(\alph*)]
	\item Sea $b$ $\in$ $\mathbb{R}$ tal que $b$ $\geq$ 2. Demostrar que $n$ $\leq$ $b^n$ para todo $n$ $\in$ $\mathbb{N}$, por inducci\'on en $n$.
  \begin{align}
		P(n) &\equiv n \leq 2^n \ \forall n \in \mathbb{N} \\
		$c.b. $P(1) &\equiv 1 \leq 2^1 \equiv True \\
    $sup. $P(n) &\equiv n \leq b^n \\
    $q.v.q. $P(n + 1) &\equiv n + 1 \leq b^{n+1} \\
    P(n + 1) &\equiv n + 1 \leq b^n \times b \\
    n \leq b^n &\implies n \times b \leq b^n \times b \\
    n + 1 &\leq n \times b \ \forall n, b \geq 2 \\
		n + 1 &\leq b^{n+1} \\
    P(n) &\implies P(n + 1) \ \ \ \blacksquare
  \end{align}

  \item Sea $b$ $\in$ $\mathbb{R}$ tal que $b$ $\geq$ 2. Demostrar que $x$ $\leq$ $b^{x+1}$ para todo $x$ $\in$ $\mathbb{R}_{\geq 0}$.
  \begin{align}
    x \leq \ &\lfloor x \rfloor + 1 \\
		$Usamos (a) con $n = &\lfloor x \rfloor + 1 \in \mathbb{N} \\
		&\lfloor x \rfloor + 1 \leq b^{\lfloor x \rfloor + 1} \\
		$Ademas $x \leq &\lfloor x \rfloor + 1 \leq x + 1 \\
		x \leq \ &\lfloor x \rfloor + 1 \leq b^{\lfloor x \rfloor + 1} \leq b^{x + 1} \\
    \implies x &\leq b^{x + 1} \ \ \ \blacksquare
  \end{align}

  \item Sean $b \in \mathbb{R}$ tal que $b$ $>$ 1 y $k \in \mathbb{N}$ tal que $b^k \geq 2$. Demostrar que $\frac{x}{k} \leq b^{x+k}$ para todo $x \in \mathbb{R}_{\geq 0}$.
	\begin{align}
		&\frac{x}{k} \leq b^{k(\frac{x}{k} + 1)} \\
		$Sabemos que $b^k &\geq 2$, renombramos $b^k = b \\
		$Hacemos lo mismo con $ &\frac{x}{k} $ ya que $ x \in \mathbb{R}_{\geq 0}, \ k \in \mathbb{N}$, renombramos $ \frac{x}{k} = x \\
		$Se cumple que $b \in &\mathbb{R} \geq 2$ y $x \in \mathbb{R}_{\geq 0}$ (hip\'otesis de (b))$ \\
		&x \leq b^{x+1}$ (vale por (b))$ \ \ \ \blacksquare
	\end{align}

  \item Sean $b \in \mathbb{R}$ tal que $b$ $>$ 1 y $k \in \mathbb{N}$ tal que $b^k \geq 2$. Demostrar que $\forall x \in \mathbb{R}_{\geq 0}$ y $\forall n,p \in \mathbb{N}$ vale la siguiente desigualdad: $(\frac{x}{pk})^n \leq b^{n(\frac{x}{p} + k)}$.
	\begin{align}
		&P(n) \equiv \left(\frac{x}{pk}\right)^n \leq b^{n(\frac{x}{p} + k)} \\
		&$c.b. Tomamos $\frac{x}{p} = j, j \in \mathbb{R}_{\geq 0} \\
		&P(1) \equiv \frac{j}{k} \leq b^{j + k}$ (vale por (c))$ \\
		&$sup. $P(n) \\
		&$q.v.q. $P(n + 1) \equiv \left(\frac{x}{pk}\right)^{n + 1} \leq b^{(n+1)(\frac{x}{p} + k)}
	\end{align}
	Veamos que evaluar P(n + 1) es lo mismo que multiplicar ambos lados por sus mismos factores, entonces si la desigualdad se cumplia, al multiplicar ambos lados por lo mismo se va a seguir cumpliendo. \ \ \ $\blacksquare$

  \item Demostrar que, vistas como funciones de $x \in \mathbb{R}_{\geq 0}$, vale: $x^p \in O(b^x)$ para toda base $b \in \mathbb{R}$ tal que $b > 1$ y para todo exponente $p \in \mathbb{N}$.
	\begin{align}
		&f(x) = x^{p} \land g(x) = b^{x} \\
		&$q.v.q. $(\exists x_{0} \in \mathbb{N}, \ c \in \mathbb{R}_{\geq 0}) \ f(x) \leq c.g(x) \ \forall x \geq x_{0} \\
		&$ Usamos (d) y tomamos $n = p \\
		&\left(\frac{x}{pk}\right)^p \leq b^{p(\frac{x}{p}+k)} \\
		&$Tomamos $\frac{x}{p} = x, x \in \mathbb{R}_{\geq 0} \\
		&\frac{x^p}{k^p} \leq b^{px} \times b^{pk} \\
		&x^p \leq b^{px} \times b^{pk} \times k^p \\
		&$Veamos que aca podemos tomar $k^p$ y $b^{pk}$ como constantes$ \\
		&x^p \leq b^{p^{x}} \times c_{1} \times c_{2} \\
		&$Tomamos $b^p = b \\
		&x^p \leq b^{x} \times c \\
		&$Tomando $x_{0} = x$ vale que $x^p \in O(b^x) \ \ \ \blacksquare
	\end{align}
\end{enumerate}

\textbf{Ejercicio 2: Complejidades}
\begin{codesnippet}
\begin{verbatim}
function P(A : arreglo(nat)) -> B : arreglo(nat) {
  n := tam(A)
  M := 0
  for i := 0 to n - 1 {
    if (A[i] >= n) { A[i] := 0 } else { M := max(M, A[i]) }
  }
  B := nuevo arreglo(nat) desde 0 hasta M inclusive, inicializado en 0
  for i := 0 to M {
    for j := i to M {
      B[A[i]] := 1 + B[A[i]] + B[A[j]]
    }
  }
  return B
}
\end{verbatim}
\end{codesnippet}

En esta funcion, tenemos 2 variables importantes
\begin{enumerate}
  \item n: Tama\~no del arreglo A
  \item M: Elemento mas grande en el arreglo A, menor a long(A). Est\'a acotado. (0 $\leq$ M $\leq$ n - 1)
\end{enumerate}

Sabemos que la primer iteracion va a ser $\Omega$(n) $\land$ O(n) = $\Theta$(n) ya que no importa que valores tenga n, siempre se va a iterar todo el tama\~no del arreglo A.
\\ \\
Generar el nuevo arreglo B inicializado en 0 es $\Theta$(M) pero va a estar acotado por la segunda iteracion.
\\ \\
Ademas, la segunda iteracion depende del M.
\begin{enumerate}
  \item Mejor caso: Ocurre cuando el segundo ciclo no se ejecuta nunca, es decir que el elemento m\'as grande en el arreglo A sea 0, o que A sea una lista con todos ceros. A = [0, 0, 0 ... , 0].
  $$\sum_{i=0}^{M=0} 1 = 0 = \Omega(0)$$
  \item Peor caso: el elemento m\'as grande del arreglo A es igual a n - 1 entonces vamos a tener M = n - 1. Reemplazando el M por su peor caso para calcular la complejidad nos da una sumatoria ya que en el primer for iteramos de 0 a M, y en el segundo desde i hasta M.
  $$\sum_{i=0}^{M} (M - i) = \sum_{i=0}^{n-1} (n - 1 - i) = \sum_{i=0}^{n - 1} i = \frac{n \times (n - 1)}{2} = \frac{n^2 - n}{2} = O(n^2)$$
\end{enumerate}

Sumamos los ordenes de mejor y peor caso de cada iteracion

$\implies$ P $\in$ $\Omega$(n) $\land$ P $\in$ O(n^2)
\\

\textbf{Ejercicio 3: Editor de Texto}
\begin{enumerate}[label=\alph*)]
	\item Escribir en castellano el invariante de representaci\'on
	\begin{enumerate}[label=\arabic*.]
		\item La pesta\~na seleccionada no puede estar en el conjunto de las pesta\~nas vac\'ias ni en el conjunto de las no vac\'ias.
		\item Las pesta\~nas vac\'ias no pueden estar en las no vac\'ias y viceversa.
		\item Todas las pesta\~nas estan definidas en alg\'un conjunto (vac\'ias, no vac\'ias o seleccionada).
		\item Las pesta\~nas no vac\'ias no pueden tener su texto vac\'io.
	\end{enumerate}
	\item Escribir formalmente el invariante de representaci\'on \\ \\
	Pasemos los predicados a l\'ogica de primer orden.
	\begin{enumerate}[label=(\arabic*)]
		\item $e.seleccionada \notin e.inactivasVac\'ias \ \land$ \\ $(\forall t: tupla<nat, string>)(t \in e.inactivasNoVac\'ias \implies \pi_{1}(t) \neq e.seleccionada)$
		\item $(\forall i: nat)(\forall t: tupla<nat, string>)((i \in e.inactivasVac\'ias \land t \in e.inactivasNoVac\'ias) \implies i \neq \pi_{1}(t))$
		\item $(\forall i: nat)(0 \leq i \leq long(e.inactivasVac\'ias) + long(e.inactivasNoVac\'ias) $\\$ \implies i = e.seleccionada \lor i \in e.inactivasVac\'ias \ \lor$ \\ $(\exists t: tupla<nat, string>)(t \in e.inactivasNoVac\'ias \land \pi_{1}(t) = i))$
		\item $(\forall t: tupla<nat, string>)(t \in e.inactivasNoVac\'ias \implies \neg vac\'ia?(\pi_{2}(t)))$
	\end{enumerate} \\
	Luego, nuestro invariante de representaci\'on queda de la forma
	\begin{center}
		Rep: $\widehat{estr} \rightarrow boolean$ \\
		Rep($e$) $\equiv (1) \land (2) \land (3) \land (4)$
	\end{center}
	\item Escribir formalmente la funci\'on de abstracci\'on
	\begin{center}
		Abs: $\widehat{estr} \ e \rightarrow$ Editor \{Rep($e$)\} \\
		$(\forall e: \widehat{estr})$ Abs($e$) $=_{obs}$ $ed$: Editor $\mid$

		$\#pesta\~nas(ed) = long(e.inactivasVac\'ias) + long(e.inactivasNoVac\'ias) + 1 \ \land_{L}$

		$(\forall i: nat)(0 \leq i \leq \#pesta\~nas(ed) \impluego seleccionada?(ed, \ i) = true \Leftrightarrow e.seleccionada = i) \ \land$

		$(\forall i: nat)(i \in e.inactivasVac\'ias \impluego texto(ed, i) = \ <>) \ \land$

		$(\forall t: tupla<nat, string>)(t \in e.inactivasNoVac\'ias \impluego texto(ed, \ \pi_{1}(t)) = \pi_{2}(t)) \ \land$

		$texto(ed, \ e.seleccionada) = e.anteriores \circ e.posteriores \ \land_{L}$

		$posicionCursor(ed) = long(e.anteriores)$
	\end{center}

\begin{itemize}
	\item La cantidad de pesta\~nas es la suma de las inactivas + 1.
	\item La seleccionada es la que esta en e.seleccionada.
	\item Los textos de las vac\'ias son vac\'ios, los de las no vac\'ias son su segundo valor de la tupla y el de la seleccionada es la suma de los anteriores con los posteriores.
	\item La posicion del cursor es la longitud de los caracteres anteriores.
\end {itemize}
\end{enumerate}

\end{document}
